\documentclass{article}\usepackage[]{graphicx}\usepackage[]{color}
%% maxwidth is the original width if it is less than linewidth
%% otherwise use linewidth (to make sure the graphics do not exceed the margin)
\makeatletter
\def\maxwidth{ %
  \ifdim\Gin@nat@width>\linewidth
    \linewidth
  \else
    \Gin@nat@width
  \fi
}
\makeatother

\definecolor{fgcolor}{rgb}{0.345, 0.345, 0.345}
\newcommand{\hlnum}[1]{\textcolor[rgb]{0.686,0.059,0.569}{#1}}%
\newcommand{\hlstr}[1]{\textcolor[rgb]{0.192,0.494,0.8}{#1}}%
\newcommand{\hlcom}[1]{\textcolor[rgb]{0.678,0.584,0.686}{\textit{#1}}}%
\newcommand{\hlopt}[1]{\textcolor[rgb]{0,0,0}{#1}}%
\newcommand{\hlstd}[1]{\textcolor[rgb]{0.345,0.345,0.345}{#1}}%
\newcommand{\hlkwa}[1]{\textcolor[rgb]{0.161,0.373,0.58}{\textbf{#1}}}%
\newcommand{\hlkwb}[1]{\textcolor[rgb]{0.69,0.353,0.396}{#1}}%
\newcommand{\hlkwc}[1]{\textcolor[rgb]{0.333,0.667,0.333}{#1}}%
\newcommand{\hlkwd}[1]{\textcolor[rgb]{0.737,0.353,0.396}{\textbf{#1}}}%
\let\hlipl\hlkwb

\usepackage{framed}
\makeatletter
\newenvironment{kframe}{%
 \def\at@end@of@kframe{}%
 \ifinner\ifhmode%
  \def\at@end@of@kframe{\end{minipage}}%
  \begin{minipage}{\columnwidth}%
 \fi\fi%
 \def\FrameCommand##1{\hskip\@totalleftmargin \hskip-\fboxsep
 \colorbox{shadecolor}{##1}\hskip-\fboxsep
     % There is no \\@totalrightmargin, so:
     \hskip-\linewidth \hskip-\@totalleftmargin \hskip\columnwidth}%
 \MakeFramed {\advance\hsize-\width
   \@totalleftmargin\z@ \linewidth\hsize
   \@setminipage}}%
 {\par\unskip\endMakeFramed%
 \at@end@of@kframe}
\makeatother

\definecolor{shadecolor}{rgb}{.97, .97, .97}
\definecolor{messagecolor}{rgb}{0, 0, 0}
\definecolor{warningcolor}{rgb}{1, 0, 1}
\definecolor{errorcolor}{rgb}{1, 0, 0}
\newenvironment{knitrout}{}{} % an empty environment to be redefined in TeX

\usepackage{alltt}
\usepackage{hyperref}

\title{SOP 84: Obtaining Climate Records}
\author{Marc Los Huertos}
\IfFileExists{upquote.sty}{\usepackage{upquote}}{}
\begin{document}
\maketitle

\section{Terrestrial Meteorological Data}


\subsection{Climate Data Online (CDO) system}

There are several ways to obtain online climate data and the CDO-web interace is probably the easiest, using the following URL: \url{https://www.ncdc.noaa.gov/cdo-web/}. 

Climate data, such as temperature and precipitation are recorded by station -- so this is generally the observation unit. Even if you don't have a location of interest nailed down, this interface can help you find sites and evaluate the length of their records. 

Below is a description of the steps to get data from a site:

\begin{description}
  \item[Follow Link] Browse Datasets
  \item[Expand Link] Daily Summaries
  \item[Change Date Range] Select the earliest data on the list to the present. Although your site might not have these records, it's easier to set the greatest range upfront.
  \item[Add Search Terms] Select the state or country of interest.
  \item[Start Search] Once you have initiated a search, the next window will give a list of records -- click in these records to see what they might be composed of. Then decide which station you will want to evaluate.
  \item[Add to Cart] Don't worry, there is no charge (yet!)
  \item[Station Detail \& Data Flag Options] Change to metric, dataset to include ppt and temperature.
  \item[Click 'Continue' Button]
  
  NOTE: Common Errors:
  
Error -- *Text order size is 3,948 Station Years, which exceeds our capacity of 1,000 Station Years. Please select fewer stations/locations, or reduce the date range.

If you get this error, you need to reduce the number of stations selected.
  
  \item[Complete Order] Type in your email address and you will get a confirmation relativley soon and the data in a day or so...
\end{description}

\subsection{Historical Climatology Network-Daily (GHCN‐D) data}


\url{https://www1.ncdc.noaa.gov/pub/data/cdo/documentation/GHCND_documentation.pdf}

\url{ftp://ftp.ncdc.noaa.gov/pub/data/ghcn/daily/gsn/}

\subsection{rNOAA library tools}

There is a relatively new tool to access NOAA data using an R package, called rNOAA. 

To use the tool, several dependent packages need to be installed and loaded, including rNOAA. Next users need to get an API code, which is quite simple.

Using the package is relatively strait foreward, although I have been getting timeout errors that I can't solve. 

Here's a URL for a tutorial: \url{https://ropensci.org/tutorials/rnoaa_tutorial.html}.

First, we prepare R by loading two libraries that we'll use to download (rNOAA) and manipulate (plyr) the data.
\begin{knitrout}
\definecolor{shadecolor}{rgb}{0.969, 0.969, 0.969}\color{fgcolor}\begin{kframe}
\begin{alltt}
\hlkwd{library}\hlstd{(}\hlstr{'plyr'}\hlstd{)}
\hlkwd{library}\hlstd{(}\hlstr{'rnoaa'}\hlstd{)}
\end{alltt}
\end{kframe}
\end{knitrout}

\subsubsection{List of Cities}

rNOAA has a simple function to list the cities:
\begin{knitrout}
\definecolor{shadecolor}{rgb}{0.969, 0.969, 0.969}\color{fgcolor}\begin{kframe}
\begin{alltt}
\hlkwd{ncdc_locs}\hlstd{(}\hlkwc{locationcategoryid}\hlstd{=}\hlstr{'CITY'}\hlstd{,} \hlkwc{sortfield}\hlstd{=}\hlstr{'name'}\hlstd{,} \hlkwc{sortorder}\hlstd{=}\hlstr{'desc'}\hlstd{,} \hlkwc{token} \hlstd{=} \hlstr{'qZqZPeprQLtooYJMiFzCEqqaNMdGJRgb'}\hlstd{)}
\end{alltt}


{\ttfamily\noindent\bfseries\color{errorcolor}{\#\# Error in curl::curl\_fetch\_memory(url, handle = handle): Timeout was reached}}\end{kframe}
\end{knitrout}

\subsubsection{Getting Data}
\begin{knitrout}
\definecolor{shadecolor}{rgb}{0.969, 0.969, 0.969}\color{fgcolor}\begin{kframe}
\begin{alltt}
\hlkwd{ncdc_stations}\hlstd{(}\hlkwc{datasetid}\hlstd{=}\hlstr{'GHCND'}\hlstd{,} \hlkwc{locationid}\hlstd{=}\hlstr{'FIPS:12017'}\hlstd{,} \hlkwc{stationid}\hlstd{=}\hlstr{'GHCND:USC00084289'}\hlstd{,} \hlkwc{token} \hlstd{=} \hlstr{'qZqZPeprQLtooYJMiFzCEqqaNMdGJRgb'}\hlstd{)}
\end{alltt}


{\ttfamily\noindent\bfseries\color{errorcolor}{\#\# Error in curl::curl\_fetch\_memory(url, handle = handle): Timeout was reached}}\end{kframe}
\end{knitrout}

\subsection{Evaluating Records}

\subsection{Export Options}

\section{Sea Surface Temperature Data}

\section{Satellite Data}

\section{Ice-Core Data}



\end{document}
