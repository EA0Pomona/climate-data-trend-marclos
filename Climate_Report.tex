\documentclass{article}\usepackage[]{graphicx}\usepackage[]{color}
%% maxwidth is the original width if it is less than linewidth
%% otherwise use linewidth (to make sure the graphics do not exceed the margin)
\makeatletter
\def\maxwidth{ %
  \ifdim\Gin@nat@width>\linewidth
    \linewidth
  \else
    \Gin@nat@width
  \fi
}
\makeatother

\definecolor{fgcolor}{rgb}{0.345, 0.345, 0.345}
\newcommand{\hlnum}[1]{\textcolor[rgb]{0.686,0.059,0.569}{#1}}%
\newcommand{\hlstr}[1]{\textcolor[rgb]{0.192,0.494,0.8}{#1}}%
\newcommand{\hlcom}[1]{\textcolor[rgb]{0.678,0.584,0.686}{\textit{#1}}}%
\newcommand{\hlopt}[1]{\textcolor[rgb]{0,0,0}{#1}}%
\newcommand{\hlstd}[1]{\textcolor[rgb]{0.345,0.345,0.345}{#1}}%
\newcommand{\hlkwa}[1]{\textcolor[rgb]{0.161,0.373,0.58}{\textbf{#1}}}%
\newcommand{\hlkwb}[1]{\textcolor[rgb]{0.69,0.353,0.396}{#1}}%
\newcommand{\hlkwc}[1]{\textcolor[rgb]{0.333,0.667,0.333}{#1}}%
\newcommand{\hlkwd}[1]{\textcolor[rgb]{0.737,0.353,0.396}{\textbf{#1}}}%
\let\hlipl\hlkwb

\usepackage{framed}
\makeatletter
\newenvironment{kframe}{%
 \def\at@end@of@kframe{}%
 \ifinner\ifhmode%
  \def\at@end@of@kframe{\end{minipage}}%
  \begin{minipage}{\columnwidth}%
 \fi\fi%
 \def\FrameCommand##1{\hskip\@totalleftmargin \hskip-\fboxsep
 \colorbox{shadecolor}{##1}\hskip-\fboxsep
     % There is no \\@totalrightmargin, so:
     \hskip-\linewidth \hskip-\@totalleftmargin \hskip\columnwidth}%
 \MakeFramed {\advance\hsize-\width
   \@totalleftmargin\z@ \linewidth\hsize
   \@setminipage}}%
 {\par\unskip\endMakeFramed%
 \at@end@of@kframe}
\makeatother

\definecolor{shadecolor}{rgb}{.97, .97, .97}
\definecolor{messagecolor}{rgb}{0, 0, 0}
\definecolor{warningcolor}{rgb}{1, 0, 1}
\definecolor{errorcolor}{rgb}{1, 0, 0}
\newenvironment{knitrout}{}{} % an empty environment to be redefined in TeX

\usepackage{alltt}

\title{Climate Narratives Report Help}
\author{Leah Ho-Israel}
\IfFileExists{upquote.sty}{\usepackage{upquote}}{}
\begin{document}

\maketitle

\section{Process}
It is sometimes useful to conduct a literature review at multiple points in your report writing process. This is because data collection and experiemental design are necessarily based on previous results and literature.

For this project, use figures "that you need to tell your story." There is little mention of outside figures in the project guidelines, but you should use them only as needed. 

In order to help make your report more specific, mention that there are multiple areas of effect from climate change and summarize the effect, but you may want to choose only one or two areas to focus on. For example, public health, environmental, economic, industry/agricultural. 
In terms of data analysis, you have already run TMin, and many of you are beginning to look at specific months. This is a great idea to also help narrow your report. It may also help to group months by season, and look at temperature change as compared to seasonality.

\section{Literature Review}
\begin{itemize}
  \item Gives good overview of local climate history
  \item Includes region-specific information—industry, agriculture, geographical, ecological/environmental, population (growth/demographics), income level, sensitivity to climate change, public involvement in/awareness of climate change/likelihood to get involved
  \item Mentions historical climate record/important climate regions in area.
\end{itemize}
Sample A
\begin{quote}
Wisconsin generally has cold and snowy winters, and the Great Lakes are usually frozen over at least through February. Historically temperatures are above 32.2 degrees Celsius (90 degrees Fahrenheit) less than 14 times a year. Because of its climate, it has a high level of dairy farming and vegetable farming, especially corn, hay, and soybeans (University of Wisconsin, 2013. Additionally, many cold-water species of fish, like trout, inhabit the lakes and streams. Wisconsin has a lot of winter recreational activities, including cross-country skiing, snowmobiling, ice fishing, snowshoeing, hockey, and more (University of Wisconsin, 2013). In the future, the climate is expected to warm, along with various other effects noted later on. The future climate of Green Bay, which is directly east from Marshfield, is expected to become similar to the current climate of Reeses Mill, West Virginia, far southeast of Wisconsin. Change in mean average temperature is supposed to increase significantly as well (Veloz et al., 2012). 
\end{quote}


\end{document}
