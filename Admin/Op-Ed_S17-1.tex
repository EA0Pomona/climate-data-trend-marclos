\section{Topic: Why Should We Care about Temperature Changes?}

\subsection{Rationale}

Climate change may be the most controversial environmental issue in history. However, compared to other issues, this history is relatively short. Fueled by opposing political parties and industry goals, the conclusions of scientists is a fundamental source of conflict -- thus, science itself has become extremely politicized. 

Nevertheless, how and where science and scientists became embroiled became a battle ground negotiating the appropriate level of regulation (regulatory reach), economic and industrial \textit{Laissez-faire}, and environmental risks. Environmental issues are almost always controversial and in the case of climate change, few dominate the political agenda like climate change. 

\subsection{Assignment}

Write an Op-Ed piece that outlines why residents in a specific US region should care about temperature changes. Spend sometime deciding what is currently in the news that you consider a compelling issue to your audience.

\subsection{Due Date}

Submit your Op Ed via \texttt{Sakai} by the Wednesday, January 25th, 5 pm.

\subsection{What is compelling?}

I will be grading these according several criteria. First, the topic must be compelling -- connecting current affairs to the historical issues of climate. Second, I will be looking for good use of evidence and citations, while creating fluid prose that compel the reader to continue reading. If the reader gets stuck in statistics, it can be like wading in mud -- but without some ``numbers'' the argument may become glittering generalities without a sense of a gritty reality. Again, your job is to find a compelling balance. Finally, you want the read to jump out of their seat and ``do something''. Thus, the Op-Ed should compel the reader into action, see assignment handout for more information.

%\subsection{Readings}




