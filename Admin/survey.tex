\documentclass[a4paper,10pt,BCOR10mm,oneside,headsepline]{scrartcl}\usepackage[]{graphicx}\usepackage[]{color}
%% maxwidth is the original width if it is less than linewidth
%% otherwise use linewidth (to make sure the graphics do not exceed the margin)
\makeatletter
\def\maxwidth{ %
  \ifdim\Gin@nat@width>\linewidth
    \linewidth
  \else
    \Gin@nat@width
  \fi
}
\makeatother

\definecolor{fgcolor}{rgb}{0.345, 0.345, 0.345}
\newcommand{\hlnum}[1]{\textcolor[rgb]{0.686,0.059,0.569}{#1}}%
\newcommand{\hlstr}[1]{\textcolor[rgb]{0.192,0.494,0.8}{#1}}%
\newcommand{\hlcom}[1]{\textcolor[rgb]{0.678,0.584,0.686}{\textit{#1}}}%
\newcommand{\hlopt}[1]{\textcolor[rgb]{0,0,0}{#1}}%
\newcommand{\hlstd}[1]{\textcolor[rgb]{0.345,0.345,0.345}{#1}}%
\newcommand{\hlkwa}[1]{\textcolor[rgb]{0.161,0.373,0.58}{\textbf{#1}}}%
\newcommand{\hlkwb}[1]{\textcolor[rgb]{0.69,0.353,0.396}{#1}}%
\newcommand{\hlkwc}[1]{\textcolor[rgb]{0.333,0.667,0.333}{#1}}%
\newcommand{\hlkwd}[1]{\textcolor[rgb]{0.737,0.353,0.396}{\textbf{#1}}}%
\let\hlipl\hlkwb

\usepackage{framed}
\makeatletter
\newenvironment{kframe}{%
 \def\at@end@of@kframe{}%
 \ifinner\ifhmode%
  \def\at@end@of@kframe{\end{minipage}}%
  \begin{minipage}{\columnwidth}%
 \fi\fi%
 \def\FrameCommand##1{\hskip\@totalleftmargin \hskip-\fboxsep
 \colorbox{shadecolor}{##1}\hskip-\fboxsep
     % There is no \\@totalrightmargin, so:
     \hskip-\linewidth \hskip-\@totalleftmargin \hskip\columnwidth}%
 \MakeFramed {\advance\hsize-\width
   \@totalleftmargin\z@ \linewidth\hsize
   \@setminipage}}%
 {\par\unskip\endMakeFramed%
 \at@end@of@kframe}
\makeatother

\definecolor{shadecolor}{rgb}{.97, .97, .97}
\definecolor{messagecolor}{rgb}{0, 0, 0}
\definecolor{warningcolor}{rgb}{1, 0, 1}
\definecolor{errorcolor}{rgb}{1, 0, 0}
\newenvironment{knitrout}{}{} % an empty environment to be redefined in TeX

\usepackage{alltt}
\usepackage[ngerman]{babel}
\usepackage[utf8]{inputenc}
\usepackage{wasysym}% provides \ocircle and \Box
\usepackage{enumitem}% easy control of topsep and leftmargin for lists
\usepackage{color}% used for background color
\usepackage{forloop}% used for \Qrating and \Qlines
\usepackage{ifthen}% used for \Qitem and \QItem
\usepackage{typearea}
\areaset{17cm}{26cm}
\setlength{\topmargin}{-1cm}
\usepackage{scrpage2}
\pagestyle{scrheadings}
\ihead{Project II: Topic Selection Survey}
\ohead{\pagemark}
\chead{}
\cfoot{}

%%%%%%%%%%%%%%%%%%%%%%%%%%%%%%%%%%%%%%%%%%%%%%%%%%%%%%%%%%%%
%% Beginning of questionnaire command definitions %%
%%%%%%%%%%%%%%%%%%%%%%%%%%%%%%%%%%%%%%%%%%%%%%%%%%%%%%%%%%%%
%%
%% 2010, 2012 by Sven Hartenstein
%% mail@svenhartenstein.de
%% http://www.svenhartenstein.de
%%
%% Please be warned that this is NOT a full-featured framework for
%% creating (all sorts of) questionnaires. Rather, it is a small
%% collection of LaTeX commands that I found useful when creating a
%% questionnaire. Feel free to copy and adjust any parts you like.
%% Most probably, you will want to change the commands, so that they
%% fit your taste.
%%
%% Also note that I am not a LaTeX expert! Things can very likely be
%% done much more elegant than I was able to. If you have suggestions
%% about what can be improved please send me an email. I intend to
%% add good tipps to my website and to name contributers of course.
%%
%% 10/2012: Thanks to karathan for the suggestion to put \noindent
%% before \rule!

%% \Qq = Questionaire question. Oh, this is just too simple. It helps
%% making it easy to globally change the appearance of questions.
\newcommand{\Qq}[1]{\textbf{#1}}

%% \QO = Circle or box to be ticked. Used both by direct call and by
%% \Qrating and \Qlist.
\newcommand{\QO}{$\Box$}% or: $\ocircle$

%% \Qrating = Automatically create a rating scale with NUM steps, like
%% this: 0--0--0--0--0.
\newcounter{qr}
\newcommand{\Qrating}[1]{\QO\forloop{qr}{1}{\value{qr} < #1}{---\QO}}

%% \Qline = Again, this is very simple. It helps setting the line
%% thickness globally. Used both by direct call and by \Qlines.
\newcommand{\Qline}[1]{\noindent\rule{#1}{0.6pt}}

%% \Qlines = Insert NUM lines with width=\linewith. You can change the
%% \vskip value to adjust the spacing.
\newcounter{ql}
\newcommand{\Qlines}[1]{\forloop{ql}{0}{\value{ql}<#1}{\vskip0em\Qline{\linewidth}}}

%% \Qlist = This is an environment very similar to itemize but with
%% \QO in front of each list item. Useful for classical multiple
%% choice. Change leftmargin and topsep accourding to your taste.
\newenvironment{Qlist}{%
\renewcommand{\labelitemi}{\QO}
\begin{itemize}[leftmargin=1.5em,topsep=-.5em]
}{%
\end{itemize}
}

%% \Qtab = A "tabulator simulation". The first argument is the
%% distance from the left margin. The second argument is content which
%% is indented within the current row.
\newlength{\qt}
\newcommand{\Qtab}[2]{
\setlength{\qt}{\linewidth}
\addtolength{\qt}{-#1}
\hfill\parbox[t]{\qt}{\raggedright #2}
}

%% \Qitem = Item with automatic numbering. The first optional argument
%% can be used to create sub-items like 2a, 2b, 2c, ... The item
%% number is increased if the first argument is omitted or equals 'a'.
%% You will have to adjust this if you prefer a different numbering
%% scheme. Adjust topsep and leftmargin as needed.
\newcounter{itemnummer}
\newcommand{\Qitem}[2][]{% #1 optional, #2 notwendig
\ifthenelse{\equal{#1}{}}{\stepcounter{itemnummer}}{}
\ifthenelse{\equal{#1}{a}}{\stepcounter{itemnummer}}{}
\begin{enumerate}[topsep=2pt,leftmargin=2.8em]
\item[\textbf{\arabic{itemnummer}#1.}] #2
\end{enumerate}
}

%% \QItem = Like \Qitem but with alternating background color. This
%% might be error prone as I hard-coded some lengths (-5.25pt and
%% -3pt)! I do not yet understand why I need them.
\definecolor{bgodd}{rgb}{0.8,0.8,0.8}
\definecolor{bgeven}{rgb}{0.9,0.9,0.9}
\newcounter{itemoddeven}
\newlength{\gb}
\newcommand{\QItem}[2][]{% #1 optional, #2 notwendig
\setlength{\gb}{\linewidth}
\addtolength{\gb}{-5.25pt}
\ifthenelse{\equal{\value{itemoddeven}}{0}}{%
\noindent\colorbox{bgeven}{\hskip-3pt\begin{minipage}{\gb}\Qitem[#1]{#2}\end{minipage}}%
\stepcounter{itemoddeven}%
}{%
\noindent\colorbox{bgodd}{\hskip-3pt\begin{minipage}{\gb}\Qitem[#1]{#2}\end{minipage}}%
\setcounter{itemoddeven}{0}%
}
}

%%%%%%%%%%%%%%%%%%%%%%%%%%%%%%%%%%%%%%%%%%%%%%%%%%%%%%%%%%%%
%% End of questionnaire command definitions %%
%%%%%%%%%%%%%%%%%%%%%%%%%%%%%%%%%%%%%%%%%%%%%%%%%%%%%%%%%%%%
\IfFileExists{upquote.sty}{\usepackage{upquote}}{}
\begin{document}

\begin{center}
\textbf{\huge Project II: Topic Selection Survey}
\end{center}\vskip1em

\noindent Using this survey, I will try to accomplish two things: 1) Align your interests with agriculture so that you can dive deeper into a particular type of system, 2) Provide a way for us to have student expertise that we can learn from each week.

\section*{About you}

\Qitem{ \Qq{Your name}: \Qline{8cm} }

\Qitem{ \Qq{Where are you from?} I am from \Qline{3.5cm} (city, state).}

\Qitem{ \Qq{Is it important to understand types of agriculture associated with a region that you are invested?} \hskip0.4cm \QO{}
absolutely \hskip1.5cm \QO{} not really because: \Qline{4cm} }

\Qitem{ \Qq{Do you like having an option to select a crop type?}
\begin{Qlist}
\item Yes, this meets my need for autonomy.
\item No, I'd rather have someone else decide for me.
\item Not sure.
\item It doesn't matter that much to me.
\end{Qlist}
}

\minisec{Please rate the following choices}
\vskip.5em

\Qitem[a]{ \Qq{Cereals} \Qtab{3cm}{less interested \Qrating{5} super interested}}

\Qitem[b]{ \Qq{Specialty Crops} \Qtab{3cm}{less interested \Qrating{5} super interested}}

\Qitem[c]{ \Qq{Potential for Change} \Qtab{3cm}{less interested \Qrating{5} super interested}}

\Qitem[d]{ \Qq{Highly entrenched} \Qtab{3cm}{less interested \Qrating{5} super interested}}

\section*{About this questionnaire}

\Qitem{\Qq{Do you like the style so far?} \Qtab{5.5cm}{\QO{} Yes
\hskip0.5cm \QO{} No}}

\Qitem{\Qq{Is it really worth the ink?} \Qtab{5.5cm}{\QO{} Guess so.
\hskip0.5cm \QO{} Probably not. \hskip0.5cm \QO{} Don't know.}}

\Qitem{\Qq{How does this item look different from the previous one?}
\Qtab{10.5cm}{\QO{}\Qtab{0.6cm}{Oh. Does it?}}\par
\Qtab{10.5cm}{\QO{}\Qtab{0.6cm}{No clue. I just can't figure it out.
So sorry.}}\par
\Qtab{10.5cm}{\QO{}\Qtab{0.6cm}{I guess what you mean is that here
the different answer options are below each other.}}\par
}

\Qitem[a]{ \Qq{Please describe your first impression.} \Qlines{2} }

\Qitem[b]{ \Qq{In case you would like some more lines to write, here
they are:} \Qlines{4} }

\section*{Another feature}

\noindent Now, here is another nice feature: you can automatically
alter the background color of the items. Here is how it can look like.
Because we are too lazy to think of new questions we will just ask you
the same questions again. There is one difference to the above though,
can you find it?

\renewcommand{\QO}{$\ocircle$}% Use circles now instead of boxes.

\section*{About you}

\QItem{ \Qq{Your name}: \Qline{8cm} }

\QItem{ \Qq{How old are you?} I am \Qline{1.5cm} years old.}

\QItem{ \Qq{Are you in a good mood right now?} \hskip0.4cm \QO{}
absolutely \hskip0.5cm \QO{} not really because: \Qline{3cm} }

\QItem{ \Qq{Do you like having an option?}
\begin{Qlist}
\item Yes, this meets my need for autonomy.
\item No, I'd rather have someone else decide for me.
\item Not sure.
\end{Qlist}
}

\QItem{ \Qq{What kind of music do you like best?}
\begin{Qlist}
\item Johann Sebastian Bach
\item Jazz
\item The Beatles
\item Other: \Qline{4cm}
\end{Qlist}
}

\minisec{Please evaluate the following composers}
\vskip.5em

\QItem[a]{ \Qq{Mozart} \Qtab{3cm}{horrible \Qrating{5} fantastic}}

\QItem[b]{ \Qq{Beethoven} \Qtab{3cm}{horrible \Qrating{5} fantastic}}

\QItem[c]{ \Qq{Johann S. Bach} \Qtab{3cm}{horrible \Qrating{5} fantastic}}

\QItem[d]{ \Qq{John Lennon}\Qtab{2.5cm}{\parbox[t]{3.3cm}{\raggedleft
Uh, well, I do not like his music very much}
\Qrating{7} \parbox[t]{3cm}{Oh, yes, you know, really great
stuff}}}

\QItem[e]{ \Qq{Elvis}\Qtab{2.5cm}{\parbox[t]{3.3cm}{\raggedleft
Uh, well, I do not like his music very much}
\Qrating{7} \parbox[t]{3cm}{Oh, yes, you know, really great
stuff}}}

\section*{About this questionnaire}

\QItem{\Qq{Do you like the style so far?} \Qtab{5.5cm}{\QO{} Yes
\hskip0.5cm \QO{} No}}

\QItem{\Qq{Is it really worth the ink?} \Qtab{5.5cm}{\QO{} Guess so.
\hskip0.5cm \QO{} Probably not. \hskip0.5cm \QO{} Don't know.}}

\QItem{\Qq{How does this item look different from the previous one?}
\Qtab{10.5cm}{\QO{}\Qtab{0.6cm}{Oh. Does it?}}\par
\Qtab{10.5cm}{\QO{}\Qtab{0.6cm}{No clue. I just can't figure it out.
So sorry.}}\par
\Qtab{10.5cm}{\QO{}\Qtab{0.6cm}{I guess what you mean is that here
the different answer options are below each other.}}\par
}

\QItem[a]{ \Qq{Please describe your first impression.} \Qlines{2} }

\QItem[b]{ \Qq{In case you would like some more lines to write, here
they are:} \Qlines{4} }

\end{document}
