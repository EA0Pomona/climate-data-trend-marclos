% subsection{Climate Science Review}

\subsubsection{Rational}

Learning about climate science requires students to dig deep into meteorology, atmospheric chemistry, physics, biogeochemistry, etc. Thus, learning about the science requires a bit of patience and lots of hard work. 

Using a literature genre, this asssignment has been designed to give you the capacity to be familiar and develop some expertise in one topic of climate change and to become a resource to others in the class.

\subsubsection{Assignment}

Investigate the assigned topic so you can write a thoughtful and critical review of the topics. Be sure to include how data might be used to counter common arguments that critique climate change science. Submit a written summary of your research findings and their references. The paper should be less than 8 pages (double spaced). 

Please include the following sections:

\begin{itemize*}
  \item Historical Development.
  \item Data Sources and Methods of Analysis.
  \item Areas of Uncertainty.
  \item Politicization of Evidence and Conclusions;
\end{itemize*}
 
\subsubsection{Submission Format and Naming Convention}

Submit via \texttt{Sakai} using the following naming convention: Climate\_Science\_Review\_F17\_XXXXX.pdf, where the XXXXX refer to the five digit \href{https://github.com/marclos/Climate_Change_Narratives/raw/master/Admin/RandomNumbers.pdf}{assigned random numbers}.

\subsubsection{Climate Science Review Grading}

The Climate Science Review will be grading using Table \ref{tab:climatesciencereviewgrading}. 

\begin{table}[h]
\caption{Climate Science Grading Standards.}
\label{tab:climatesciencereviewgrading}
\begin{tabular}{lll}\hline
Criteria                & Standard      & Percent \\
\hline\hline
Historical Development  & Completness   & 20\% \\
Data Sources and Methods  & Completeness  & 20\% \\
Logical Fallacies & Identified and Confronted   & 20\% \\
Critical Thinking & Advanced                    & 20\%\\
% Graphics          && 20\%\\
Writing           && 20\% \\
Submission        && 20\% \\
\hline
\end{tabular}
\end{table}

