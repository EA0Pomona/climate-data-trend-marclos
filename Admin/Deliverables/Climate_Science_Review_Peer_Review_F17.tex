%\subsection{Peer Review of Climate Science Report}

\subsubsection{Rational}

Communicating about climate sciene is fraught with potential stumbling blocks. First, it's hard to hit the audience knowledge level correctly. Second, many readers have biases, which means that readers have filters that we might not be able to appreciate. Finally, since we are not climate scientists, we are working to translate the science into a languages that others can understand -- back to first point!  

By using peer reveiw, we can develop methods that might reduce this stumbling blocks, where your peers will be able to read and evaluate if the text is clear, accurate, and comprehendable. 

\subsubsection{Assignment}

Read two of your peers' Climate Report and evaluate the report using the following questions as a guide:

\begin{enumerate}
  \item What is the argument of the report?
  \item Describe three sources of evidence for this argument. What is compelling about these sources? What might be criticisms about these sources? If evidence is missing please explain.
  \item What counter arguments are used in the report? Are these accurately characterized? Is the evidence for the counter arguments documented and evaluated to your satisfaction?  If not what's missing?
  \item List terms that may not be defined well enough for you and the general public. 
  \item Re-write two sentences that could be improved for clarity.
  \item Re-write two sentences that could be improved for precision.
  \item Re-write two sentences that could be improved for accuracy.
\end{enumerate}


<<echo=F, results = 'asis', fig.cap='Peer Review Assignments'>>=
library(xtable)
Reviewer = c("Kirara","Ally", "Caroline", "Brooke","Meily", "Valentina", "Bebe")
Reviewed1 = c("11790", "NA", "61296",     "NA",   "91789",  "Valentina", "Bebe")
Reviewed2 =c("91789", "NA", "Caroline", "NA", "11790", "61269", "98159", "Bebe")
project1teams <- data.frame(Reviewer = reviewer, "To be reviewed"=presentDate)
print(xtable(project1teams), include.rownames=FALSE)
@

\subsubsection{Submission Format and Naming Convention}

Climate\_Science\_PeerReview\_XXXXX.pdf

\subsubsection{Peer Review Grading}

Table~\ref{tab:peerreviewgrading} will be used to evaluate the peer review exercise. 

\begin{table}[h]
\caption{Peer Review Grading Standards.}
\label{tab:peerreviewgrading}
\begin{tabular}{lll}\hline
Standard                      &   Percent   & \\ 
\hline\hline
Identifying Argument          &   5\%      & \\
Evidence Evaluation           &   10\%      & \\
Counter Argument Evaluation & 5\%    & \\
List of Terms               & 20\%    & \\
Re-write -- Clarity         & 20\%    & \\
Re-write -- Precision       & 20\%    & \\
Re-write -- Accuracy        & 20\%    & \\

\hline
\end{tabular}
\end{table}

