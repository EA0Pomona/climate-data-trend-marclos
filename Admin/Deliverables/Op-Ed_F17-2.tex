%\subsection{Op-Ed 2}

\subsubsection{Rational}

\subsubsection{Assignment}

\subsubsection{Submission Format and Naming Convention}

Using the Op-Ed guidelines, write an Op-Ed to summarize 2-3 salient points from your Blog where you should:

\begin{itemize}
  \item Describe regional climate changes and predictions that include ecological impacts; 
  \item Cite instances of how various scientific arguments have been distorted and politicized;
  \item Identify how conventional scientific standards have been compromised and how arguments that might be based on distortions can be countered.
\end{itemize}

Uses the Op-Ed guidelines, submit a draft Op-Ed via \texttt{Sakai}. Include a description of the local or regional papers that this Op-Ed might be submitted and several examples of Op-Eds that have discussed environmental issues.

Write an Op-Ed to propose what makes a good public product with respect to criticisms of climate science debates and criticisms. In other words, describe (2-3) ways that climate change skeptisism might misuse the data analysis and how one might prevent the misuse, be sure to cite your blog as an attepmpt to accomlish these goals. 

Submit Op-Ed to the appropriate regional or local paper.

\noindent Assignment: 
\begin{itemize*}
  \item Make an organized presentation that effectively communicates how various scientific arguments have been distorted and politicized;
  \item Identify how conventional scientific standards have been comprimised; and
  \item Use the allotted time (10 min) effectively. I suggest you practice, 10 mintues can go very quickly when presenting complext scientific data.
  
\end{itemize*}

\subsubsection{Grading of Published Op Ed 2}




