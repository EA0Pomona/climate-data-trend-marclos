% subsection{Climate Science Presentation}

\subsubsection{Rational}

\subsubsection{Assignment}

Create a presentation as a R Presentation (.Rprs) and it each person should limit their presentation to 10 minutes. Longer presentations will be penalized. Ten minutes goes quickly, so I suggesty you practice a few times to ensure that you don't lose unnecessary points. 

\noindent Assignment: 
\begin{itemize*}
  \item Make an organized presentation that effectively communicates how various scientific arguments have been distorted and politicized;
  \item Identify how conventional scientific standards have been comprimised; and
  \item Use the allotted time (10 min) effectively. I suggest you practice, 10 mintues can go very quickly when presenting complext scientific data.
\end{itemize*}

\subsubsection{Submission Format and Naming Convention}

In addition, each team will present (via open-source software, i.e. rPres) their results to the class. 
ainty.

\subsubsection{Presentation Grading Criteria}

The Climate Science Presentation will be grading using the criteria in Table \ref{tab:presentationgrading}.

\begin{table}[h]
\centering
\caption{Presentation Grading Criteria}
\label{tab:presentationgrading}
\begin{tabular}{ll}\hline
Standard            & Percent \\ \hline\hline    
Accuracy            & 20\%  \\
Completeness        & 20\% \\
Clarity             & 20\% \\
Timeliness          & 20\% \\
Use of Technology   & 20\% \\ \hline
\end{tabular}
\end{table}