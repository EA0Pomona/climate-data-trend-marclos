% subsection{Climate Science Presentation}

\subsubsection{Rational}

Climate change science is complex and requires a tacit understanding of a range of scientific disciplines. Instead of trying to learn all about them, we will hear presentations from our peers on various topics based on their own research. 

Following the adage, 'the best way to learn is to teach', is an appropriate way to think about this assignment. 

\subsubsection{Assignment}

Create a presentation as a R Presentation (.Rprs) and it each person should limit their presentation to 8 minutes/person. Eight minutes goes quickly, so I suggesty you practice a few times to ensure that you don't lose unnecessary points. Longer presentations will be penalized.  

\noindent Assignment: 
\begin{itemize*}
  \item Describe the historical development of the scinece/topic.
  \item Describe how data are collected and used to develop conclusions.
  \item Describe areas of uncertainty.
  \item Make an organized presentation that effectively communicates how various scientific arguments have been distorted and politicized;
  \item Identify how conventional scientific standards have been comprimised; and
  \item Use the allotted time (8 min) effectively. I suggest you practice, 8 mintues can go very quickly when presenting complext scientific data.
\end{itemize*}

\subsubsection{Submission Format and Naming Convention}

In addition, each team will present (via open-source software, i.e. rPres) their results to the class. Learning to use Rpres is pretty easy, however, making it do things that we want is another story. 

I have created a template that you can use to copy on in the R project: Communications\_Resources in a file called: Introducing\_rPres.md. Open this file and you will find a preview button where the knit button has been in other files. 

I suggest you open this file and then save a version in your own directory (under Student\_Folders) with the following name -- ``Expert\_Topic.Md, '' where topic is some one word description of the content.

\subsubsection{Presentation Grading Criteria}

The Climate Science Presentation will be grading using the criteria in Table \ref{tab:presentationgrading}.

\begin{table}[h]
\centering
\caption{Presentation Grading Criteria}
\label{tab:presentationgrading}
\begin{tabular}{ll}\hline
Standard            & Percent \\ \hline\hline    
Accuracy            & 20\%  \\
Completeness        & 20\% \\
Clarity             & 20\% \\
Timeliness          & 20\% \\
Use of Technology   & 20\% \\ \hline
\end{tabular}
\end{table}