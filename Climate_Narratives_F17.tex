\documentclass{article}\usepackage[]{graphicx}\usepackage[]{color}
%% maxwidth is the original width if it is less than linewidth
%% otherwise use linewidth (to make sure the graphics do not exceed the margin)
\makeatletter
\def\maxwidth{ %
  \ifdim\Gin@nat@width>\linewidth
    \linewidth
  \else
    \Gin@nat@width
  \fi
}
\makeatother

\definecolor{fgcolor}{rgb}{0.345, 0.345, 0.345}
\newcommand{\hlnum}[1]{\textcolor[rgb]{0.686,0.059,0.569}{#1}}%
\newcommand{\hlstr}[1]{\textcolor[rgb]{0.192,0.494,0.8}{#1}}%
\newcommand{\hlcom}[1]{\textcolor[rgb]{0.678,0.584,0.686}{\textit{#1}}}%
\newcommand{\hlopt}[1]{\textcolor[rgb]{0,0,0}{#1}}%
\newcommand{\hlstd}[1]{\textcolor[rgb]{0.345,0.345,0.345}{#1}}%
\newcommand{\hlkwa}[1]{\textcolor[rgb]{0.161,0.373,0.58}{\textbf{#1}}}%
\newcommand{\hlkwb}[1]{\textcolor[rgb]{0.69,0.353,0.396}{#1}}%
\newcommand{\hlkwc}[1]{\textcolor[rgb]{0.333,0.667,0.333}{#1}}%
\newcommand{\hlkwd}[1]{\textcolor[rgb]{0.737,0.353,0.396}{\textbf{#1}}}%
\let\hlipl\hlkwb

\usepackage{framed}
\makeatletter
\newenvironment{kframe}{%
 \def\at@end@of@kframe{}%
 \ifinner\ifhmode%
  \def\at@end@of@kframe{\end{minipage}}%
  \begin{minipage}{\columnwidth}%
 \fi\fi%
 \def\FrameCommand##1{\hskip\@totalleftmargin \hskip-\fboxsep
 \colorbox{shadecolor}{##1}\hskip-\fboxsep
     % There is no \\@totalrightmargin, so:
     \hskip-\linewidth \hskip-\@totalleftmargin \hskip\columnwidth}%
 \MakeFramed {\advance\hsize-\width
   \@totalleftmargin\z@ \linewidth\hsize
   \@setminipage}}%
 {\par\unskip\endMakeFramed%
 \at@end@of@kframe}
\makeatother

\definecolor{shadecolor}{rgb}{.97, .97, .97}
\definecolor{messagecolor}{rgb}{0, 0, 0}
\definecolor{warningcolor}{rgb}{1, 0, 1}
\definecolor{errorcolor}{rgb}{1, 0, 0}
\newenvironment{knitrout}{}{} % an empty environment to be redefined in TeX

\usepackage{alltt}
%\documentclass{tufte-handout}
\usepackage{hyperref}

\newenvironment{itemize*}%
  {\begin{itemize}%
    \setlength{\itemsep}{0pt}%
    \setlength{\parskip}{0pt}}%
  {\end{itemize}}
	
\newenvironment{enumerate*}%
  {\begin{enumerate}%
    \setlength{\itemsep}{0pt}%
    \setlength{\parskip}{0pt}}%
  {\end{enumerate}}

\title{Do weather changes matter?}
\author{Marc Los Huertos}
%\date{}
\IfFileExists{upquote.sty}{\usepackage{upquote}}{}
\begin{document}

\maketitle
\tableofcontents

\section{Introduction}

\subsection{Climate and the IPCC}

According the the Inter-Governmental Panel on Climate Change or IPCC, the temperature has been changing about 0.85 degrees C since the  1880s -- but this global average is not evenly distributed accross the globe. 

This change and causes of this change are perhaps one of the most contested environmental issues in the 50 year history of environmental movement. So much so, that as EA students, we need to understand who and how these conclusions were made, while understanding the potential implications.

\subsubsection{What is the IPCC?}

First, the IPCC is the Intergovernmental Panel on Climate Change (IPCC) is a scientific and intergovernmental body under the auspices of the United Nations, set up at the request of member governments, dedicated to the task of providing the world with an objective, scientific view of climate change and its political and economic impacts.

The Intergovernmental Panel on Climate Change was created in 1988. It was set up by the World Meteorological Organization (WMO) and the United Nations Environment Program (UNEP) to prepare, based on available scientific information, assessments on all aspects of climate change and its impacts, with a view of formulating realistic response strategies. 

\subsubsection{IPCC's Role}

The role of the IPCC is to assess on a comprehensive, objective, open and transparent basis the scientific, technical and socio-economic information relevant to understanding the scientific basis of risk of human-induced climate change, its potential impacts and options for adaptation and mitigation.

As an intergovernmental body, membership of the IPCC is open to all member countries of the United Nations (UN) and WMO. Currently 195 countries are Members of the IPCC.

The IPCC has published five comprehensive assessment reports reviewing the latest climate science (Table \ref{tab:IPCC}), as well as a number of special reports on particular topics. These reports are prepared by teams of relevant researchers selected by the Bureau from government nominations. Drafts of these reports are made available for comment in open review processes to which anyone may contribute.

\begin{table}
\caption{Major IPCC Reports}\label{tab:IPCC}
\centering
\begin{tabular}{lc}\hline
Assessment Report             & Published \\ \hline\hline
First Assessment Report (FAR) & 1990    \\
Supplementary Report          & 1992   \\
Second Asessment Report (SAR) & 1995    \\
Third Assessment Report (TAR) & 2001    \\
Fourth Assessment Report (AR4)& 2007   \\
Fifth Assessment Report (AR5) & 2014  \\
Sixth Assessment Report (AR6) & 2022*  \\ \hline
\end{tabular}
\end{table}

Each assessment report is in three volumes, corresponding to Working Groups I, II, and III. Unqualified, ``the IPCC report'' is often used to mean the Working Group I report, which covers the basic science of climate change.

\subsection{Global and Regional Average Temperature Changes}

An average temperture increse for the globe is somewhat abstract and, perhaps, beyond what humans can reliable perceive. Are there strategies to help us appreciate these potential weather changes? Perhaps, we should evaluate how temperature (and/or rainfall) might be changing at regional scales. 

Thus, for this project, we'll try to understand how temperature changes ``map'' onto a community that we care about? But to do this we need obtain and analyze tempterature data and determine if weather changes have compelling impacts on local communities.

In other words, do weather changes matter?

\subsection{Goals of this Document}

\begin{enumerate*}
  \item Describe the goals and approach for the project;
  \item Provide or point to resources to prepare for and conduct the project; and
  \item Describe how we will evaluate the project process and products.
\end{enumerate*}

\section{Project Description}

\subsection{Driving Question(s)}

Projects can often be structured as questions, but sometimes it it worth phrasing the questions in a number of ways -- this might help you find ways that you might find the question more provactive and interesting, For example,

\begin{itemize*}
  \item Is my region's climate changing?
  \item How is climate change affecting my community?
\end{itemize*}

But you can modify these questions to develop the project that you might find compelling.

In addition, we may develop ``sub-questions'' that can be developed or answered that might inform the main question or questions. For example, 

\begin{itemize*}
  \item Are there biases in weather data? Can these biases be corrected? If so, how?
  \item How can we evaluate trends? What are the most appropriate statistical tools to test for trends?
  \item What is the best way to display visual data?  Are there best practices to guide a public product to make it more compelling or interactive?
\end{itemize*}

\subsection{Public Products}

Science is a social project. From the questions we ask, to the results and their presentation, science is embedded in a culture of norms. To frame our science within these norms, each of us will publish blogs to answer the question, ``do weather changes matter?''

In addition, each student will write and submit an OpEd piece to a regional newspaper that frames regional climate issues into a newsworthy item.

Finally, we will hold a Q \&A session with public school teachers to help them implement NGSS standards on weather and climate.

\section{Directed Practice}

\subsection{Learning Goals}

For this project, you will use weather data to the question ``do weather changes matter.'' How you answer the question is largely up to you, however, to be successful students will demonstration compentency in some specific skills and knowledge. 

\paragraph{Skills}

\begin{itemize*}
  \item Ability to download and process weather data;
  \item evaluate temporal trends in weather data;
  \item research the environmental impacts on human or non-human communities; and
  \item communicate conclusions to the public with special attention to guide how data misinterpretations should be considered.
\end{itemize*}

\paragraph{Knowledge}
\begin{itemize*}
  \item Understand how data climate data is currated;
  \item Analyze climate impacts from around the world.
\end{itemize*}

Throughout this project, your team and instructor will develop the strategies and skills to address this question and help you make some conclusions and present the results ot the public.

\subsection{Resources}

Students will have the following tools available:

\begin{itemize*}
  \item Servers where stored weather data can be downloaded;
  \item R Studio Server with some scripts \& libraries to help develop analyses;
  \item Gighub to store project codes and as a platform to make the product public;
  \item Lectures, reports, and presentations on climate change science, the social and ecological implications of climate change, and the polcies and politics of climate change;
  \item \href{https://github.com/marclos/Climate_Change_Narratives/raw/master/Admin/RandomNumbers.pdf}{Random numbers for student submissions}; and
  \item Shiny app templates that might be used as a container for interactive content.\footnote{Currently under-development -- We will likely skip this application since I not confident in using this particular tool.}
\end{itemize*}

\subsubsection{Software Guides}

Much of the environmental data collected has become electronic. 

liberation nd freedom...

\href{https://github.com/marclos/Climate_Change_Narratives/raw/master/Admin/Liberation_via_Open_Source_Software.pdf}{Open Source and Liberation}


Thus, to access to and process these data, we need use tools to access, pre-process, and analyze these data using computer software. Below are resources that we have developed to assist you in this class (Table \ref{tab:softwareguides}).

\begin{table}[h]
\caption{Software guides developed for EA30/31. These SOPs have been developed by students and faculty over the years and are loaded on the github.com/SOPs repository.}\label{tab:softwareguides}
\centering
\begin{tabular}{ll}\hline
SOP \#    & Description                                 \\\hline\hline
06        & An Introduction to R, Rstudio, Github \\
06b       & Introduction to Markdown--Html  \\
06c       & Introduction to Markdown--Word  \\
XXx       & Visual Display of Data using R  \\
\hline
\end{tabular}
\end{table}

\subsubsection{Data Processing and Analysis Tools}

Much of the environmental data collected has become electronic. Thus, to access to and process these data, we need use tools to access, pre-process, and analyze these data using computer software. 

Below are resources that we have developed to assist you in this class (Table \ref{tab:tools}).

\begin{table}[h]

\caption{Resources to obtain, pre-process, and analyze NOAA climate data.}\label{tab:tools}
\centering
\begin{tabular}{ll}\hline
SOP \#    & Description                                 \\\hline\hline
84        & \href{https://github.com/marclos/Climate_Change_Narratives/raw/master/Analysis_SOPs/SOP84_Obtaining_Climate_Records.pdf}{Obtaining Climate Records}\\
85        & \href{https://github.com/marclos/Climate_Change_Narratives/raw/master/Analysis_SOPs/SOP85_Using_NOAA_Climate_Records.pdf}{Using NOAA climate Records}\\
90        & \href{https://github.com/marclos/Climate_Change_Narratives/raw/master/Analysis_SOPs/SOP90_Analyzing_Trends.pdf}{Analyzing Climate Trends} \\ \hline
\end{tabular}
\end{table}

These SOPs can be found in the Rproject/Github Respistory --Climate Change Narratives and in the 'Analysis\_SOPs' directory.

The analysis of trend data can range from simple to complex. For a brief introduction, read an introduction on the \href{https://climatedataguide.ucar.edu/climate-data-tools-and-analysis/trend-analysis}{Trend Analysis} on the Climate Data Guide website.

\subsubsection{Readings and Other Climate Change Resources}

\subsubsection{Contested Science and Critical Thinking}


\begin{itemize*}
  \item \href{https://github.com/marclos/Climate_Change_Narratives/raw/master/Communication_Resources/Logical_Fallacies.pdf}{``The Rhetorical Tools of Logical Fallacies''}
  
  \item \href{https://github.com/marclos/Climate_Change_Narratives/raw/master/Communication_Resources/Critical_Thinking.pdf}{``Critical Thinking in EA''}
\end{itemize*}

\subsubsection{Communication Resources}

We will learn and practice our skills to communicate using written and oral media. 

Scientific writing is a skill that takes years to develop. Although there are many types of readings, scientific writing does have some unique characteristics that will seem a bit awkward. However, you might be surprised about how much you already know about technical writing. We have selected key resources that we think will help you further develop and improve your writing skills.

Below is my list of key areas to be cognizant to improve our capacity to communicate science:

\begin{description}
  \item[Clarity, Forthrightness, and Economical]
  \item[Accuracy and Precision] Accuracy and precision occurs at several scales in writing, word choice, sentence level, paragraph, and essay level. 
  \item[Critical Thinking]
  \item[Cited Evidence]
\end{description}

However, specific genres require specific adjustments in our writing style. Please use the following to help in your writing process:

\begin{itemize*}
  \item \href{https://github.com/marclos/Climate_Change_Narratives/raw/master/Communication_Resources/Writing_About_Climate.pdf}{``Scientific Writing and Climate Narratives''}

  \item \href{https://github.com/marclos/Climate_Change_Narratives/raw/master/Communication_Resources/Op-Ed_Guidelines.pdf}{``Op-Ed Guidelines''}
  
  \item \href{https://github.com/marclos/Climate_Change_Narratives/raw/master/Communication_Resources/Scientific_Blog_Guidelines.pdf}{``Scientific Blog Guidelines''}
  
  
  \item \href{https://github.com/marclos/Climate_Change_Narratives/raw/master/Communication_Resources/Visualing_Data.pdf}{``Visual Presentation of Data using R''}
  
  \item \href{https://github.com/marclos/Climate_Change_Narratives/raw/master/Communication_Resources/Citing_Sources.pdf}{``Citing References in EA30''}
  
  \item \href{https://github.com/marclos/Climate_Change_Narratives/raw/master/Communication_Resources/Peer_Review-Dos_and_Donts.pdf}{``Peer review writing -- Dos and Don'ts''}
\end{itemize*}

Oral presentations will also be part of this project and course. Students will use Rpres for their presentations and here is a short tutorial for this tool:

\begin{itemize*}
  \item \href{https://github.com/marclos/Climate_Change_Narratives/raw/master/Communication_Resources/TBD.pdf}{``Using Rpres to Develop Oral Presentations''}
  
  \item \href{https://github.com/marclos/Climate_Change_Narratives/raw/master/Communication_Resources/TBD.pdf}{``Guide for Oral Presentations''}
\end{itemize*}

\section{Project Milestones}

To complete the project in a timely fashion, we will be adhering to a rather strick schedule (Table \ref{tab:milestones}).

\begin{table}[h]

\caption{Project Deliverables, milestones and point distribution.}\label{tab:milestones}
\begin{tabular}{lcll}\hline
Deliverable                     & Launch    & Due Date  & \% \\\hline\hline
Op-Ed \#1                   & Aug 30    & Sept 4    & 5 \\
Climate Science Report      & Sept 4    & Sept 10   & 5 \\
Climate Science Presentation& Sept 4    & Sept 11   & 5 \\
Climate Science -- Peer Review  & Sept 11 & Sept 15 & 5 \\
Draft Regional Analysis     & Sept 11   & Sept 11   & 5 \\
Regional Climate Review     & Sept 11   & Sept 22   & 10 \\
Regional Review -- Peer Review & Sept 23   & Sept 25 & 10 \\
Blog DRAFT                  & Sept 18   & Sept 29    & 15 \\
Blog --Peer Review          & Sept 30   & Oct 2     & 5\\
Blog FINAL                  & Oct 2     & Oct 8    & 20 \\
Op Ed \#2 Draft             & Oct 9     & Oct 15    & 10 \\
Op Ed \#2 Submission        & Oct 15    & Oct 20    & 5 \\ \hline
\end{tabular}
\end{table}

\section{Op Ed \#1: Why Care about Climate?}

%\subsection{Op-Ed 1}

\subsection{Rationale}

Climate change may be the most controversial environmental issue in history. However, compared to other issues, this history is relatively short. Fueled by opposing political parties and industry goals, the conclusions of scientists is a fundamental source of conflict -- thus, science itself has become extremely politicized. 

Nevertheless, how and where science and scientists became embroiled became a battle ground negotiating the appropriate level of regulation (regulatory reach), economic and industrial \textit{Laissez-faire}, and environmental risks. Environmental issues are almost always controversial and in the case of climate change, few dominate the political agenda like climate change. 

\subsection{Assignment}

Write an Op-Ed piece that outlines why residents in a specific US region should care about temperature changes. Spend sometime deciding what is currently in the news that you consider a compelling issue to your audience.

\subsection{Submission Format and Naming Convention}

Submit your Op Ed as a pdf via \texttt{Sakai}, using the following naming convention: OpEd\_XXXXX.pdf, using one of your 5 digit random numbers for the Xs. See \url{https://github.com/marclos/Climate_Change_Narratives/raw/master/Admin/RandomNumbers.pdf} to get the list of assigned random numbers. 

\subsection{Grading}

I will be grading these according several criteria. First, the topic must be compelling -- connecting current affairs to the historical issues of climate. Second, I will be looking for good use of evidence and citations, while creating fluid prose that compel the reader to continue reading. If the reader gets stuck in statistics, it can be like wading in mud -- but without some ``numbers'' the argument may become glittering generalities without a sense of a gritty reality. Again, your job is to find a compelling balance. Finally, you want the read to jump out of their seat and ``do something''. Thus, the Op-Ed should compel the reader into action, see assignment handout for more information.

%\subsection{Readings}






\section{Developing Specialized Knowledge}

To develop expertise, we will rely on teams of students to develop and evaluate various aspect of climate data. Each of us form an essential component for the effort. Organized as teams and expert groups, we will disassemble the project into chunks that each of us will contribute in specific and effective ways. This expertise will be used to develop our Q \& A sessions, as well as, to help us develop and write our op-ed and blogs. The experts should include areas of contravery and how scientists and non-scientists wressle over the data.

\subsection{Topics of Expertise}

We will will create expert groups on to present the following topics:

\begin{enumerate*}
  \item Radiative Gases -- What are they and what do they do?
  
List the major compounds categorized as radiative gases and describe how various processes determine their role as GHGs. Provide detail on how different wavelengths of light interact with the gases. Finally, a discussion of water is key, since it is one of the main sources of controversy. 
  
  \item GHG Emission Trends and Sources -- Carbon Dioxode (CO$_2$), Nitrous Oxide (N$_2$O), and Methane (CH$_4$).

Describe how carbon dioxide and other GHGs are emitted and remain in the atmosphere. Distinguish between natural and anthropogenic sources and why that distinction might be important. Desribe various type of sources and how these might be linked to certain types of economic development and activities. In addition, describe the role of vegetation and other forms of carbon sequestration.

  \item Role of Water and Other Feedbacks
  
Climate change feedback is important in the understanding of global warming because feedback processes may amplify or diminish the effect of each climate forcing, and so play an important part in determining the climate sensitivity and future climate state. Feedback in general is the process in which changing one quantity changes a second quantity, and the change in the second quantity in turn changes the first. Positive feedback amplifies the change in the first quantity while negative feedback reduces it. Be sure to include the following feedbacks: Clouds, gas release, ice-albedo, carbon, and water vapor.

  \item Terrestial Surface Temperature Records
  
The instrumental temperature record provides the temperature of Earth's climate system from the historical network of in situ measurements of surface air temperatures and ocean surface temperatures. Data are collected at thousands of meteorological stations, buoys and ships around the globe. The longest-running temperature record is the Central England temperature data series, that starts in 1659. The longest-running quasi-global record starts in 1850. In recent decades more extensive sampling of ocean temperatures at various depths have begun allowing estimates of ocean heat content but these do not form part of the global surface temperature datasets.
  
  \item Ocean Tempertures and Sea Level

Describe how ocean temperatures have been measured over time and how these have lead to a range of interpretations of the results. Disucss how the thermal expansion of water may influence sea leval rise. Discuss how sea temperature change may affect different parts of the world differently. Describe the methods to distinguish sea level rise and coastal elevation changes, including how satellites work to collect these data.  

  \item Satellite-based Temperature Measures
  
Satellites can be used to measure outgoing radition. However, each atmospheric layer has different properties and is impacted by GHGs in differing ways. Describe how the satelite data has been used, how these instruments have changed and why there are several different methods to evalaute satellite data. Because satellite data has result results, describe how these methods have been used to support or limit our confidence in climate change.
  
  \item Weather Extremes Trends Explained
  
Weather and climate extremes such as hurricanes, tornadoes, heavy downpours, heat waves, and droughts affect all sectors of the economy and the environment, impacting people where they live and work.  
  
\end{enumerate*}

\subsection{Expert Teams}

Although most of the work will be individual, we will also work in pair for the presentation. 

The following students have been assigned to the teams below:

% latex table generated in R 3.3.1 by xtable 1.8-2 package
% Wed Jul 12 19:11:24 2017
\begin{table}[ht]
\centering
\begin{tabular}{llll}
  \hline
Topic & Team\_A & Team\_B & Presentation\_Date \\ 
  \hline
1 & Brooke & Caroline & 09/11/17 \\ 
  2 & Mina & Kihara & 09/11/17 \\ 
  3 & Troy & Sarah & 09/11/17 \\ 
  4 & Kyle & Chris & 09/18/17 \\ 
  5 & Bebe & Katherine & 09/18/17 \\ 
  6 & Meily & Marc & 09/18/17 \\ 
   \hline
\end{tabular}
\end{table}


\subsection{Climate Science Presentation}

% subsection{Climate Science Presentation}

\subsubsection{Rational}

\subsubsection{Assignment}

\subsubsection{Submission Format and Naming Convention}

In addition, each team will present (via open-source software, i.e. rPres) their results to the class. The presentation shall include the following:

\begin{itemize*}
  \item Historical development of the field/issue and methods;
  \item summary of IPCC information regarding the topic;
  \item List of researchers that have contributed to this topic in the last 20 years and a description of their contribution; and
  \item Describe existing areas of uncertainty.
\end{itemize*}

Create a presentation as a R Presentation (.Rprs) and it each person should limit their presentation to 10 minutes.Longer presentations will be penalized. Ten minutes goes quickly, so I suggesty you practice a few times to ensure that you don't lose unnecessary points. 

\begin{table}[h]
\centering
\caption{Presentation Grading Criteria--F2017}
\label{tab:presentationgrading}
\begin{tabular}{ll}\hline
Standard            & Percent \\ \hline\hline    
Accuracy            & 20\%  \\
Completeness        & 20\% \\
Clarity             & 20\% \\
Timeliness          & 20\% \\
Use of Technology   & 20\% \\ \hline
\end{tabular}
\end{table}



\subsection{Climate Science Report}

% subsection{Climate Science Report}

\subsubsection{Rational}

\subsubsection{Assignment}

\subsubsection{Submission Format and Naming Convention}

Submit a written summary of your research findings and their references. Be sure to include how data might be use to counter common arguments that critique climate change science. These summaries must be loaded on the Rstudio project page and in \texttt{.Rnw} and \texttt{.pdf} formats. 

Submit via \texttt{Sakai} using the following naming convention: Climate\_Science\_Report\_F17\_XXXXX.pdf, where the XXXXX refer to the five digit \href{https://github.com/marclos/Climate_Change_Narratives/raw/master/Admin/RandomNumbers.pdf}{assigned random numbers}.

\subsubsection{Grading of Climate Science Report}

\begin{table}[h]
\caption{Climate Science Grading Standards.}
\label{tab:climatesciencereportgrading}
\begin{tabular}{lll}\hline
Standard      &   Percent   & \\ \hline\hline
\hline
\end{tabular}
\end{table}



\subsection{Peer Review of Climate Science Report}

%\subsection{Peer Review of Climate Science Report}

\subsubsection{Rational}

\subsubsection{Assignment}

\subsubsection{Submission Format and Naming Convention}

\subsubsection{Grading of Peer Review}


\begin{table}[h]
\caption{Peer Review Grading Standards.}
\label{tab:peerreviewgrading}
\begin{tabular}{lll}\hline
Standard      &   Percent   & \\ \hline\hline
\hline
\end{tabular}
\end{table}



\section{Regional Climate Analysis}

Each of us will select a region of interest. Perhaps, somewhere that you have spent a compelling time in or that you wish to know more about. Please select a region that has not been done by previous classes. 

\subsection{Analysis of Regional Data}

%\subsection{Analysis of Regional Data}

\subsubsection{Rationale}

Learning to analyze data requires a range of skills that include collecting, analyzing, and interpreting data. For our purposes, this portion of the class is what might traditionally understood as ``doing science.'' We will learn how to test a hypothesis and what it means if we reject the null hypothesis. We will create figures that can be used to communicate our results and finally, we interpret the results. 

Ultimately, this analysis will be used a template for our blogs and inform our second Opinion Editorials. 

\subsubsection{Assignment}

Using the resources supplied, it will be up to you to download, pre-process, and analyze a trend analysis using R -- where the slope, r$^2$, and probability are calculated\footnote{We will have to learn what these are to be able to explain our results! Be sure to ask lots of questions about the statistics so you appreciate this important topic that nearly every scientific field relies!} and explained. 

Using R studio, analyze a long-term climate record, create 3-4 figures that will be used to communicate these climate records, e.g. 100-year temperature \textbf{and} precipitation record for a specific region. Be sure to include language about the ``null'' hypothesis for your trend analysis.

\begin{enumerate*}
  \item Download and analyze data (i.e. make inferences) to create an public product; I have uploaded all the climate data on a network drive, \url{//fargo/classes/EA30-LosHuertos}{//fargo/classes/EA30-LosHuertos}.\footnote{I haven't been able to get the directory working consistently, so stay tuned on this.}
  
\end{enumerate*}
 that describes the methods (data sources), data quality, and trends. 


\subsubsection{Submission Format and Naming Convention}

As specified by the milestones (Table \ref{tab:milestones}), submit the draft analysis and results using Rstudio. 

The Rmd file (and the compiled html) should be saved the the 'student\_submissions' directory using the following naming convention:

Region\_XXXXX.Rmd and Region\_XXXXX.html


\medskip \noindent where XXXXX refer to one of your random numbers.

NOTE: Be sure the file still compiles. For example, you may need to change the path to the Data directory. 

Since the regional analysis has been down within Rstudio, you will use the version control procedures to commit and push your analysis onto the Github repository. Thus, be sure to commit and push your files so I have access to the files. 

\subsubsection{Data Analysis Grading}

The Data Analysis html files will be grading using the criteria in Table \ref{tab:datagrading}.

\begin{table}[h]
\caption{Summary of Data Analysis grading standards.}
\label{tab:datagrading}
\begin{tabular}{llc}\hline
Criteria            &   Standard    & Percent \\ \hline\hline
Records  & Compelling, e.g. Over 60 years & 10\% \\
Knowledge of Data & Limitations and Methods of Collection & 10\% \\
Analysis & p-values and $R^2$ reported  & 20\% \\
Analysis          & Validated Model     & 20\% \\
Interpretation    & Accurate, e.g. rejected null   & 10\% \\
Graphics          & Publishable Quality & 20\% \\
Accessible        & Pushed and named correctly  10\% \\
\hline
\end{tabular}
\end{table}

\subsection{Regional Climate Impacts -- Literature Review}

%\subsection{Regional Climate Impacts -- Literature Review}

\subsubsection{Rational}

\subsubsection{Assignment}

Review regionally relative results and conclusions from peer reviewed climate science. See this document as a resource.

Evaluate peer-reviewed articles to determine potential ecological, economic, and sociological implications of climate patterns.

Summarize these papers into a stand-alone paper. 

\subsubsection{Submission Format and Naming Convention}

The paper should be double-space, 12 point font, and less than 8 pages (excluding citations). As a pdf, the paper should be submitted via Sakai with the following naming convention:

\medskip
RegionalImpacts\_F17\_XXXXX.pdf

\medskip \noindent where the XXXXX refer to one set of the assigned random numbers. 

\subsubsection{Grading of the Regional Impacts Summary}

The regional impacts review will be grading using the criteria in Table \ref{tab:regionalimpactsgrading}.

\begin{table}[h]
\caption{Summary of Data Analysis grading standards.}
\label{tab:regionalimpactsgrading}
\begin{tabular}{llc}\hline
Criteria            &   Standard    & Percent \\ \hline\hline
Sources     & Compelling, e.g. Over 20 papers & 15\% \\
Ecological  & Knowns and unknowns             & 20\% \\
Economic    & costs and benefits              & 20\% \\
Social      & e.g. Social Justice             & 20\% \\
Communication    & Accurate, e.g. rejected null   & 25\% \\

\hline
\end{tabular}
\end{table}

\section{Communicating Science}

\subsection{Analyzing Prior Communication Edeavors}

\subsubsection{Climate Change Blogs and Websites}

\href{https://marclos.github.io/Climate_Change_Narratives/}{EA 30 Blogs}

Here are some good examples of climate blogs:

\begin{itemize*}
  \item \href{http://www.accuweather.com/en/weather-blogs/climatechange}{Accuweather}
  \item \href{http://blogs.nature.com/climatefeedback/}{Nature Magazine}
  \item \href{https://thinkprogress.org/tagged/climate}{Think Progress}
  \item \href{http://climateofourfuture.org/}{Climate Four Future}
\end{itemize*}

Useful sites: 

\begin{itemize*}
  \item \href{http://www.climatecentral.org/news/the-heat-is-on}{Climate Central}
  \item 
\end{itemize*}

\subsubsection{Climate Blogs/Websites Evaluation}

%\subsubsection{Evaluating Climate Blogs/Websites}

\subsubsection{Rational}

\subsubsection{Assignment}

\subsubsection{Submission Format and Naming Convention}

Review previously written \href{https://marclos.github.io/Climate_Change_Narratives/}{EA 30 Blogs} to evaluate which ones are effective and what you like about each one. 

Select 4-5 blogs and write a summary for each one, describe three things that you like about each one and describe one thing you might improve. Finally, look up one topic for each one that you are more interested in learning and summarize what you find.

\subsection{Writing a Scientific Blog}

%\subsection{Writing a Scientific Blog}

\subsubsection{Rational}

\subsubsection{Assignment}

\subsubsection{Submission Format and Naming Convention}

Write blog to effectively and clearly describe results.

The blog shall be publish-ready and include the following: 

\begin{itemize*}
  %\item Appropriate and thoughtful statistical analysis;
  \item Describe the economic, cultural, and physical geography of the region (2-3 sentences);
  \item Describe climate patterns (1-2 sentances);
  \item Describe where the data were obtained and summarize how the data were processed and analzyed;
  \item Time series plots of temperture data using R (3-4 graphs, with several setences describing the results);
  \item Evaluation of data to determine if trends exists;
  \item Compare results to model predictions and possible ecological and economic implications to the region; 
    \item description of what the data tells about about the region, 
  \item a few short paragraphs describing how data can be interpretted; pitfalls of unintentional and intentional misinterpretations; and 
  \item narrative that describes the climate and climate implications for a community that you care about.
  %\item Describe how the data should be presented, e.g. how the data should be interpreted, and how to avoid misinterpretations that are present in the popular culture.
\end{itemize*}

\subsection{Peer Review Blogs}

%\subsection{Peer Review Blogs}

\subsubsection{Rational}

Reviewing a public product is a priviledge. And for the `reviewed' it's a gift. Thus, for each, the reviewer and reviewed, the value for the greater good is indisputable. 

As you review your collegeues work, try to keep in mind that you are promoting a better outcome and better science. In addition, pay attention to thinks that might have escaped your own process and that you find yourself saying, ``wow, that's a cool approach!''  Perhaps, you might adapt some of the things you read into your own writing!

\subsubsection{Assignment}

To assess the Blogs, each student will review two blogs and submit a evaluation form for each one. 

\subsubsection{Submission Format and Naming Convention}

Save and submit the form as a pdf, with the following naming convention -- 

``RegionReview\_XXXXX.pdf''

\subsubsection{Blog Peer Review Grading}





\subsection{Publishing Revised Blog}

%\subsection{Publishing Revised Blog}

\subsubsection{Rational}

Our capacity to publish our blogs demonstrates that our projects have value beyond our classroom. In addition, these provide a litmus test for our work -- how will the public or specific stakeholders respond to our efforts. Will they see this a valueable, value-added, or problematic?  Although we might not get immediate feedback, the process to publish our blogs gives an opportunity that would be missing if we only wrote papers for the instructor!


\subsubsection{Assignment}

Capitalizing on the regional data analysis and impact summary, create a blog that describes the patterns of clilmate change and their implications. Your final products should include:

\begin{itemize*}
  \item Effectively display climate patterns from NOAA repositories, with at least 6 decades of data. Be sure all graphics are appropriate labeled and have captions that the reader can use to intrepret the data;
  \item Analyze the data using a linear model using R (i.e. lm);
  \item Describe the methods used to obtain and analyze the data; and
  \item Evaluate peer review literature to determine potential regional impacts from climate change -- be sure to include ecological and economic impacts; 
  \item Cite instances of how various scientific arguments have been distorted and politicized;
  \item Identify how conventional scientific standards have been compromised and how arguments that might be based on distortions can be countered.
\end{itemize*}

If it helps, read the Project\_Report.pdf on the Project Site for some helpful hints.

\subsubsection{Submission Format and Naming Convention}

The Blog will be published online (via \url{Github.com}) 

\subsubsection{Published Blog Grading}

The Climate Science Reports will be grading using Table \ref{tab:bloggrading}. 

\begin{table}[h]
\caption{Climate Science Blog Grading.}
\label{tab:bloggrading}
\begin{tabular}{lll}\hline
Standard      &   Percent   & \\ \hline\hline
\hline
\end{tabular}
\end{table}



\subsection{Op-Ed 2}

%\subsection{Op-Ed 2}

\subsubsection{Rational}

Successful editorials requires two things: 1) a compelling and newsworthy opinion, 2) engagement with a controversy associated with Climate Change and 3) a mechanism to encourage the audience to read your blog. 

\subsubsection{Assignment}

Select a regional newspaper where you can submit your Op Ed. Learn the format and length allowed for a submission. 

Using the Op-Ed guidelines, write an Op-Ed to summarize 2-3 salient points from your Blog where you should:

\begin{itemize}
  \item Describe regional climate changes and predictions that include ecological impacts; 
  \item Cite instances of how various scientific arguments have been distorted and politicized;
  \item Identify how conventional scientific standards have been compromised and how arguments that might be based on distortions can be countered.
\end{itemize}

Write and submit the editorial that highlights the key aspects of climate change that you discovered and link that to a newsworthy item. Be sure to cite and provide the URL to your blog. Submit the editorial and include the 'proof of receipt' with you submssion. 


\subsubsection{Submission Format and Naming Convention}

Uses the Op-Ed guidelines, submit a draft Op-Ed via \texttt{Sakai} and a separate document that describes the local or regional paper that this Op-Ed will be submitted to and several examples of Op-Eds that have discussed environmental issues in the paper.

Write an Op-Ed to propose what makes a good public product with respect to criticisms of climate science debates and criticisms. In other words, describe (2-3) ways that climate change skeptisism might misuse the data analysis and how one might prevent the misuse, be sure to cite your blog as an attepmpt to accomlish these goals. 

Submit Op-Ed to the appropriate regional or local paper and a copy as a PDF via \texttt{Sakai} with the following naming format: LastName\_Region\_OpEd.pdf



\subsubsection{Grading of Published Op Ed 2}

The 2nd Op Ed will be grading using the criteria in Table \ref{tab:oped2grading}.

\begin{table}[h]
\centering
\caption{Published Op Ed Grading Criteria}
\label{tab:oped2grading}
\begin{tabular}{ll}\hline
Standard            & Percent \\ \hline\hline    
Accuracy            & 20\%  \\
Completeness        & 20\% \\
Clarity             & 20\% \\
Timeliness          & 20\% \\
Use of Technology   & 20\% \\ \hline
\end{tabular}
\end{table}






\clearpage
\newpage
\section{Peer Evaluation Forms}

\subsection{Literature Review--Peer Evaluation}

\bigskip
Evaluator: \rule{7cm}{0.4pt}

\bigskip

\noindent Author: \rule{7cm}{0.4pt}

\begin{enumerate}
 \setlength\itemsep{4em}
  \item Describe two items you learned.
  \item Describe one concept or fact you would like to learn in more detail.
\end{enumerate}


\begin{table}[ht!]
\caption{Please circle the best response, where one is inadequate and five is outstanding---i.e. should be teaching the topic!}
\begin{tabular}{|p{4in}|ccccc|}\hline
How clear was the presentation?     & 1 & 2 & 3 & 4 & 5 \\ \hline
Suggestions: &&&&& \\ &&&&& \\ &&&&& \\
&&&&& \\ \hline
Did the analysis seem valid?        & 1 & 2 & 3 & 4 & 5 \\ \hline
Suggestions: &&&&& \\ &&&&& \\ &&&&& \\
&&&&& \\ \hline
Was information complete enough?            & 1 & 2 & 3 & 4 & 5 \\ \hline
Suggestions: &&&&& \\ &&&&& \\ &&&&& \\
&&&&& \\ \hline
To what extent could you use this example in climate discussions?            & 1 & 2 & 3 & 4 & 5 \\ \hline
Suggestions: &&&&& \\ &&&&& \\ &&&&& \\
&&&&& \\ \hline
\end{tabular}
\end{table}

\clearpage
\newpage
\subsection{XX--Peer Evaluation}

\bigskip
Evaluator: \rule{7cm}{0.4pt}

\bigskip

\noindent Author: \rule{7cm}{0.4pt}

\begin{enumerate}
 \setlength\itemsep{4em}
  \item Describe two items you learned.
  \item Describe one concept or fact you would like to learn in more detail.
\end{enumerate}


\begin{table}[ht!]
\caption{Please circle the best response, where one is inadequate and five is outstanding---i.e. should be teaching the topic!}
\begin{tabular}{|p{4in}|ccccc|}\hline
How clear was the presentation?     & 1 & 2 & 3 & 4 & 5 \\ \hline
Suggestions: &&&&& \\ &&&&& \\ &&&&& \\
&&&&& \\ \hline
Did the analysis seem valid?        & 1 & 2 & 3 & 4 & 5 \\ \hline
Suggestions: &&&&& \\ &&&&& \\ &&&&& \\
&&&&& \\ \hline
Was information complete enough?            & 1 & 2 & 3 & 4 & 5 \\ \hline
Suggestions: &&&&& \\ &&&&& \\ &&&&& \\
&&&&& \\ \hline
To what extent could you use this example in climate discussions?            & 1 & 2 & 3 & 4 & 5 \\ \hline
Suggestions: &&&&& \\ &&&&& \\ &&&&& \\
&&&&& \\ \hline
\end{tabular}
\end{table}

\clearpage
\newpage
\subsection{DRAFT Blog -- Peer Evaluation}

\bigskip
Evaluator: \rule{7cm}{0.4pt}

\bigskip

\noindent Presenter: \rule{7cm}{0.4pt}

\begin{enumerate}
 \setlength\itemsep{4em}
  \item Describe two items you learned.
  \item Describe one concept or fact you would like to learn in more detail.
\end{enumerate}


\begin{table}[ht!]
\caption{Please circle the best response, where one is inadequate and five is outstanding---i.e. should be teaching the topic!}
\begin{tabular}{|p{4in}|ccccc|}\hline
How clear was the presentation?     & 1 & 2 & 3 & 4 & 5 \\ \hline
Suggestions: &&&&& \\ &&&&& \\ &&&&& \\
&&&&& \\ \hline
Did the analysis seem valid?        & 1 & 2 & 3 & 4 & 5 \\ \hline
Suggestions: &&&&& \\ &&&&& \\ &&&&& \\
&&&&& \\ \hline
Was information complete enough?            & 1 & 2 & 3 & 4 & 5 \\ \hline
Suggestions: &&&&& \\ &&&&& \\ &&&&& \\
&&&&& \\ \hline
To what extent could you use this example in climate discussions?            & 1 & 2 & 3 & 4 & 5 \\ \hline
Suggestions: &&&&& \\ &&&&& \\ &&&&& \\
&&&&& \\ \hline
\end{tabular}
\end{table}



\end{document}
