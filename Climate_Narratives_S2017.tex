\documentclass{article}\usepackage[]{graphicx}\usepackage[]{color}
%% maxwidth is the original width if it is less than linewidth
%% otherwise use linewidth (to make sure the graphics do not exceed the margin)
\makeatletter
\def\maxwidth{ %
  \ifdim\Gin@nat@width>\linewidth
    \linewidth
  \else
    \Gin@nat@width
  \fi
}
\makeatother

\definecolor{fgcolor}{rgb}{0.345, 0.345, 0.345}
\newcommand{\hlnum}[1]{\textcolor[rgb]{0.686,0.059,0.569}{#1}}%
\newcommand{\hlstr}[1]{\textcolor[rgb]{0.192,0.494,0.8}{#1}}%
\newcommand{\hlcom}[1]{\textcolor[rgb]{0.678,0.584,0.686}{\textit{#1}}}%
\newcommand{\hlopt}[1]{\textcolor[rgb]{0,0,0}{#1}}%
\newcommand{\hlstd}[1]{\textcolor[rgb]{0.345,0.345,0.345}{#1}}%
\newcommand{\hlkwa}[1]{\textcolor[rgb]{0.161,0.373,0.58}{\textbf{#1}}}%
\newcommand{\hlkwb}[1]{\textcolor[rgb]{0.69,0.353,0.396}{#1}}%
\newcommand{\hlkwc}[1]{\textcolor[rgb]{0.333,0.667,0.333}{#1}}%
\newcommand{\hlkwd}[1]{\textcolor[rgb]{0.737,0.353,0.396}{\textbf{#1}}}%
\let\hlipl\hlkwb

\usepackage{framed}
\makeatletter
\newenvironment{kframe}{%
 \def\at@end@of@kframe{}%
 \ifinner\ifhmode%
  \def\at@end@of@kframe{\end{minipage}}%
  \begin{minipage}{\columnwidth}%
 \fi\fi%
 \def\FrameCommand##1{\hskip\@totalleftmargin \hskip-\fboxsep
 \colorbox{shadecolor}{##1}\hskip-\fboxsep
     % There is no \\@totalrightmargin, so:
     \hskip-\linewidth \hskip-\@totalleftmargin \hskip\columnwidth}%
 \MakeFramed {\advance\hsize-\width
   \@totalleftmargin\z@ \linewidth\hsize
   \@setminipage}}%
 {\par\unskip\endMakeFramed%
 \at@end@of@kframe}
\makeatother

\definecolor{shadecolor}{rgb}{.97, .97, .97}
\definecolor{messagecolor}{rgb}{0, 0, 0}
\definecolor{warningcolor}{rgb}{1, 0, 1}
\definecolor{errorcolor}{rgb}{1, 0, 0}
\newenvironment{knitrout}{}{} % an empty environment to be redefined in TeX

\usepackage{alltt}
%\documentclass{tufte-handout}
\usepackage{hyperref}

\newenvironment{itemize*}%
  {\begin{itemize}%
    \setlength{\itemsep}{0pt}%
    \setlength{\parskip}{0pt}}%
  {\end{itemize}}
	
\newenvironment{enumerate*}%
  {\begin{enumerate}%
    \setlength{\itemsep}{0pt}%
    \setlength{\parskip}{0pt}}%
  {\end{enumerate}}

\title{Do weather changes matter?}
\author{Marc Los Huertos}
%\date{}
\IfFileExists{upquote.sty}{\usepackage{upquote}}{}
\begin{document}

\maketitle

\section{Introduction}

According the the Inter-Governmental Panel on Climate Change or IPCC, the temperature has been changing about 0.85 degrees C since the  1880s -- but this global average is not evenly distributed accross the globe. 

How can we appreciate potential changes accross the whole globe? An average temperture increse for the globe is somewhat abstract.   Perhaps, we should evaluate how temperature (and/or rainfall) might be changing on local scales. 

Thus, for this project, we'll try to understand how temperature changes ``map'' onto a community that we care about? But to do this we need obtain and analyze tempterature data and determine if weather changes have compelling impacts on local communities.

In other words, do weather changes matter?

\subsection{Goals of this Document}

\begin{enumerate*}
  \item Describe the goals and approach for the project;
  \item Provide or point to resources to prepare for and conduct the project; and
  \item Describe how we will evaluate the project process and products.
\end{enumerate*}

\section{Project Description}

\subsection{Driving Question(s)}

Projects can often be structured as questions, but sometimes it it worth phrasing the questions in a number of ways -- this might help you find ways that you might find the question more provactive and interesting, For example,

\begin{itemize*}
  \item Is my region's climate changing?
  \item How is climate change affecting my community?
\end{itemize*}

But you can modify these questions to develop the project that you might find compelling.

In addition, we may develop ``sub-questions'' that can be developed or answered as chunks, which will be used to answer the main question or questions. For example, 

\begin{itemize*}
  \item Are there biases in weather data? Can these biases be corrected? If so, how?
  \item How can we evaluate trends? What are the most appropriate statistical tools to test for trends?
  \item What is the best way to display visual data?  Are there best practices to guide a public product to make it more compelling or interactive?
\end{itemize*}

\subsection{Public Product}

Science is a social project. From the questions we ask, to the results and their presentation, science is embedded in a culture of norms. Thus, as part of this project, students will produce a narrative blog with the following characterics:

\begin{itemize*}
  %\item Appropriate and thoughtful statistical analysis;
  \item time series data in a plot using R; 
  \item description of what the data tells about about the region, 
  \item a few short paragraphs describing how data can be interpretted; pitfalls of unintentional and intentional misinterpretations; and 
  \item narrative that describes the climate and climate implications for a community that you care about.
\end{itemize*}

In addition, we will hold a Q \&A session with public school teachers to help them implement NGSS standards on weather and climate.

\section{Approach}

Students will have the following tools available:

\begin{itemize*}
  \item Servers where stored weather data can be downloaded;
  \item R Studio Server with some scripts \& libraries to help develop analyses;
  \item Gighub to store project codes; and
  \item Student presentations on various climate change criticisms. 
  \item Shiny app templates (currently under-development) that might be used as a container for interactive content.
\end{itemize*}

\subsection{Learning Goals}

For this project, you will use weather data to the question ``do weather changes matter.'' How you answer the question is largely up to you, however, there are some learning goals associated with this project:

\paragraph{Skills}

\begin{itemize*}
  \item Ability to download and process weather data;
  \item evaluate temporal trends in weather data;
  \item evaluate environmental impacts on human or non-human communities; and
  \item communicate conclusions to the public with special attention to guide how data misinterpretations should be considered.
\end{itemize*}

\paragraph{Content}
\begin{itemize*}
  \item Understand how data climate data is currated;
  \item Analyze climate impacts from around the world.
\end{itemize*}

Throughout this project, your team and instructor will develop the strategies and skills to address this question and help you make some conclusions and present the results ot the public.

\section{Project Stages}

\subsection{Expert Groups}

To develop expertise, we will rely on teams of students to develop and evaluate various aspect of climate data. Each of us form an essential component for the effort. Organized as teams and expert groups, we will disassemble the project into chunks that each of us will contribute in specific and effective ways. This expertise will be used to develop our Q \& A sessions, as well as, to help us develop and write our op-ed and blogs.

For this project, the following students have been assigned to the teams below:

% latex table generated in R 3.3.1 by xtable 1.8-2 package
<<<<<<< HEAD
<<<<<<< HEAD
% Mon Jan 23 15:23:06 2017
=======
% Sun Jan 29 20:14:20 2017
>>>>>>> 764c27788fa815c715ffde1fd68e5a4df7724040
=======
% Mon Jan 23 15:27:11 2017
=======
% Sun Jan 29 20:14:20 2017
>>>>>>> 6fada55920d4c4f10f0e0f28a7b90dc375754a10
>>>>>>> c967d78f1d92e02884f8dbdb2538ab08e3855478
\begin{table}[ht]
\centering
\begin{tabular}{rll}
  \hline
 & Member & Team \\ 
  \hline
1 & Viraj & 1 \\ 
  2 & Kelli & 1 \\ 
  3 & Khalil & 2 \\ 
  4 & Marc & 2 \\ 
  5 & Marisa & 3 \\ 
  6 & Olivia & 3 \\ 
  7 & Claudia & 4 \\ 
  8 & Mireya & 4 \\ 
  9 & Katie & 5 \\ 
  10 & Thea & 5 \\ 
   \hline
\end{tabular}
\end{table}


We will develop expert groups on to present the following topics:

\begin{enumerate}
  \item Radiative Gases -- What are they and what do they do?
  \item GHG Emission Trends and Sources --Co2, Nitrous Oxide, and Methane.
  \item Role of Water and Other Feedbacks
  \item Ocean Tempertures and Sea level
  \item Weather Extremes Explained
\end{enumerate}

\subsection{How to Conduct a Regional Climate Analysis?}

\begin{enumerate*}

  \item Download and analyze data (i.e. make inferences) to create an public product; I have uploaded all the climate data on a network drive, \url{//fargo/classes/EA30-LosHuertos}{//fargo/classes/EA30-LosHuertos}.\footnote{I haven't been able to get the directory working consistently, so stay tuned on this.}
  
  \item Evaluate peer reviewed articles to determine potential ecological, economic, and sociological implications of climate patterns; 
  \item Write blog to effectively and clearly describe results; and
  \item Write an Op-Ed to propose what makes a good public product with respect to criticisms of climate science debates and criticisms. In other words, describe (2-3) ways that climate change skeptisism might misuse the data analysis and how one might prevent the misuse, be sure to cite your blog as an attepmpt to accomlish these goals. 
  \item Submit Op-Ed to the appropriate regional or local paper.
\end{enumerate*}

Useful sites: 

\begin{itemize}
  \item \href{http://www.climatecentral.org/news/the-heat-is-on}{Climate Central}
  \item 
\end{itemize}

\section{Stages to attain Final Product}

\subsection{Draft Op-Ed}

Uses the Op-Ed guidelines, submit a draft Op-Ed on January 25 -- please submit via \texttt{Sakai}. Include a description of the local or regional papers that this Op-Ed might be submitted and several examples of Op-Eds that have discussed environmental issues.

\subsection{Expert Team Reports}

Present a one page summary with citations that explains the topic assigned. Submit via \texttt{Sakai}. 

\subsection{Expert Team Presentations}

In addition, each team will present (via appropriate presentation software) their results to the class. 

\noindent Assignment: 
\begin{itemize}
  \item Make an organized presentation that effectively communicates how various scientific arguments have been distorted and politicized;
  \item Identify how conventional scientific standards have been comprimised; and
  \item Use the allotted time (10 min) effectively. I suggest you practice, 10 mintues can go very quickly when presenting complext scientific data.
  
\end{itemize}

\subsection{Draft Graphics and Data Analysis}

Create 3-4 figures to communicate climate records, e.g. 100-year temperature and precipitation record for a specific region. Include a trend analysis using R -- where the slope, r$^2$, and probability are calculated\footnote{We will have to learn what these are to be able to explain our results! Be sure to ask lots of questions so you appreciate this important topic that nearly every scientific field relies!} and explained. Using Rstudio, push Rmd and html files to the project Github site. 

\subsection{Summary of Literature Review}

Use digital resources to determine the implications of climate on the the region of interest. Summarize these impacts into a one or two page document that includes at least 10 peer-reviewed journal articles. Submit this via \texttt{Sakai} by the Feb. 6th. 

\subsection{Public Blog}

The blog shall be publish-ready and include the following: 

\begin{itemize}
  \item Describe the economic, cultural, and physical geography of the region;
  \item Describe climate patterns;
  \item Time series graph of temperture data from a specific region using R;
  \item Evaluation of data to determine if trends exists;
  \item Compare results to model predictions and possible ecological and economic implications to the region; 
  \item Describe how the data should be presented, e.g. how the data should be interpreted, and how to avoid misinterpretations that are present in the popular culture.
  
\end{itemize}

The Blog will be published by 5 pm, Feb. 10th.

\subsection{Oral Presentation-- Preping the Op-Ed}

Students will present a summary of their written report during the last week of classes. 

The 10 minute presentation will include the following:

\begin{itemize}
  \item Geographic description of region;
  \item Demographics and a brief history;
  \item Summary of economic geography;
  \item Available data records;
  \item Summary of data analysis;
  \item Climate model implications for the region; and
  \item Analysis of political-science debate in region.
  
\end{itemize}

\subsection{Op-Ed}

Using the Op-Ed guidelines, re-write your first Op-Ed to summarize 2-3 salient points from your Blog where you should:

\begin{itemize}
  \item Describe regional climate changes and predictions that include ecological impacts; 
  \item Cite instances of how various scientific arguments have been distorted and politicized;
  \item Identify how conventional scientific standards have been compromised and how arguments that might be based on distortions can be countered.
\end{itemize}


\section{Grading}

Your final products should include:

\begin{itemize}
  \item Effectively display climate patterns from NOAA repositories, with at least 6 decades of data. Be sure all graphics are appropriate labeled and have captions that the reader can use to intrepret the data;
  \item Analyze the data using a linear model using R (i.e. lm);
  \item Describe the methods used to obtain and analyze the data; and
  \item Evaluate peer review literature to determine potential regional impacts from climate change -- be sure to include ecological and economic impacts; 
  \item Cite instances of how various scientific arguments have been distorted and politicized;
  \item Identify how conventional scientific standards have been compromised and how arguments that might be based on distortions can be countered.
\end{itemize}

\begin{table}
\caption{Proportion of points and timeline.}
\begin{tabular}{lll}\hline
Product                     & \% of Project   & Due Date \\\hline\hline
Draft Op-Ed                 & 5\%             & Jan. 25th \\
Background Report           & 5\%             & Feb. 3rd \\
Background Presentation     & 5\%             & Feb. 6th \\
Draft Figures \& Analysis   & 15\%            & Feb. 6th \\
Summary of Literature Review& 10\%            & Feb. 10th \\
Blog                        & 20\%            & Feb. 13th \\
Pre-Op Ed Discussion        & 10\%            & Feb. 15th \\     
Op Ed                       & 20\%            & Feb. 19th \\
Q \& A Session              & 10\%            & TBD \\ \hline

\end{tabular}
\end{table}

\clearpage
\newpage
\subsection{Oral Presentation--Peer Evaluation}

\bigskip
Evaluator: \rule{7cm}{0.4pt}

\bigskip

\noindent Presenter: \rule{7cm}{0.4pt}

\begin{enumerate}
 \setlength\itemsep{4em}
  \item Describe two items you learned.
  \item Describe one concept or fact you would like to learn in more detail.
\end{enumerate}


\begin{table}[ht!]
\caption{Please circle the best response, where one is inadequate and five is outstanding---i.e. should be teaching the topic!}
\begin{tabular}{|p{4in}|ccccc|}\hline
How clear was the presentation?     & 1 & 2 & 3 & 4 & 5 \\ \hline
Suggestions: &&&&& \\ &&&&& \\ &&&&& \\
&&&&& \\ \hline
Did the analysis seem valid?        & 1 & 2 & 3 & 4 & 5 \\ \hline
Suggestions: &&&&& \\ &&&&& \\ &&&&& \\
&&&&& \\ \hline
Was information complete enough?            & 1 & 2 & 3 & 4 & 5 \\ \hline
Suggestions: &&&&& \\ &&&&& \\ &&&&& \\
&&&&& \\ \hline
To what extent could you use this example in climate discussions?            & 1 & 2 & 3 & 4 & 5 \\ \hline
Suggestions: &&&&& \\ &&&&& \\ &&&&& \\
&&&&& \\ \hline
\end{tabular}
\end{table}


\end{document}
